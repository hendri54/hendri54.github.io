\PassOptionsToPackage{unicode=true}{hyperref} % options for packages loaded elsewhere
\PassOptionsToPackage{hyphens}{url}
%
\documentclass[]{article}
\usepackage{lmodern}
\usepackage{amssymb,amsmath}
\usepackage{ifxetex,ifluatex}
\usepackage{fixltx2e} % provides \textsubscript
\ifnum 0\ifxetex 1\fi\ifluatex 1\fi=0 % if pdftex
  \usepackage[T1]{fontenc}
  \usepackage[utf8]{inputenc}
  \usepackage{textcomp} % provides euro and other symbols
\else % if luatex or xelatex
  \usepackage{unicode-math}
  \defaultfontfeatures{Ligatures=TeX,Scale=MatchLowercase}
\fi
% use upquote if available, for straight quotes in verbatim environments
\IfFileExists{upquote.sty}{\usepackage{upquote}}{}
% use microtype if available
\IfFileExists{microtype.sty}{%
\usepackage[]{microtype}
\UseMicrotypeSet[protrusion]{basicmath} % disable protrusion for tt fonts
}{}
\IfFileExists{parskip.sty}{%
\usepackage{parskip}
}{% else
\setlength{\parindent}{0pt}
\setlength{\parskip}{6pt plus 2pt minus 1pt}
}
\usepackage{hyperref}
\hypersetup{
            pdfborder={0 0 0},
            breaklinks=true}
\urlstyle{same}  % don't use monospace font for urls
\setlength{\emergencystretch}{3em}  % prevent overfull lines
\providecommand{\tightlist}{%
  \setlength{\itemsep}{0pt}\setlength{\parskip}{0pt}}
\setcounter{secnumdepth}{0}
% Redefines (sub)paragraphs to behave more like sections
\ifx\paragraph\undefined\else
\let\oldparagraph\paragraph
\renewcommand{\paragraph}[1]{\oldparagraph{#1}\mbox{}}
\fi
\ifx\subparagraph\undefined\else
\let\oldsubparagraph\subparagraph
\renewcommand{\subparagraph}[1]{\oldsubparagraph{#1}\mbox{}}
\fi

% set default figure placement to htbp
\makeatletter
\def\fps@figure{htbp}
\makeatother


\date{}

\begin{document}

\hypertarget{answer-sketch-writing-assignment-3}{%
\section{Answer Sketch: Writing Assignment
3}\label{answer-sketch-writing-assignment-3}}

\begin{enumerate}
\def\labelenumi{\arabic{enumi}.}
\item
  Higher \(i^*\) shifts AD to the right. Intuition: It causes FX
  appreciation (due to capital outflows). A positive shock to NX.

  \begin{enumerate}
  \def\labelenumii{\arabic{enumii}.}
  \item
    Note that it would be incorrect to argue that AD shifts left because
    \(i \uparrow\). The change in \(i\) is an outcome, not a shock.
    Changing \(i\) is a move along AD, not a shift.
  \end{enumerate}
\item
  MR:

  \begin{enumerate}
  \def\labelenumii{\arabic{enumii}.}
  \item
    Y unchanged. \(P \uparrow\). Both from the diagram, which looks like
    a closed economy positive AD shock. 
  \item
    From LM: \(i \uparrow\) because \(M/P \downarrow\). 
  \item
    Therefore, investment falls (crowding out). 
  \item
    From \(Y = C + I + G + NX\) it follows that \(NX \uparrow\). \(C\)
    is of course unchanged.
  \item
    We end up with higher \(i\), but still \(i < i^*\) (at \(i = i^*\),
    the dollar would be unchanged and AD would fall).
  \item
    The story: capital outflows cause the dollar to appreciate. NX
    increase; a positive demand shock. Higher AD raises prices.
    Households sell bonds to get more liquidity, so that \(i \uparrow\).
    That crowds out investment.
  \end{enumerate}
\item
  SR:

  \begin{enumerate}
  \def\labelenumii{\arabic{enumii}.}
  \item
    Fairly similar to MR, except that now \(Y \uparrow\). Also
    \(P \uparrow\) from the AS/AD diagram.
  \item
    Both imply \(i \uparrow\) from LM.
  \item
    \(C \uparrow\).
  \item
    Since \(Y - C \uparrow = I + G + NX \uparrow\), the change in \(I\)
    is ambiguous.
  \item
    The sequence of events is pretty much the same as in the MR, except
    that output is up (b/c price expectations have not yet adjusted).
  \end{enumerate}
\item
  If the Fed tries to avoid crowding out, it needs to expand the money
  supply to the point where \(i\) returns to its original value.

  \begin{enumerate}
  \def\labelenumii{\arabic{enumii}.}
  \item
    Relative to the previous cases, AD shifts out more.
  \item
    In the MR \(Y,C,I\) are all unchanged, and therefore \(NX\) is
    unchanged. The Fed can fully neutralize the change in \(i^*\). 
  \item
    This works by raising domestic prices to the point where the real
    exchange rate is back to where it started. This neutralizes the NX
    shock.
  \end{enumerate}
\item
  Here we encounter a logical hiccup in the model (because we did not
  model expectations).

  \begin{enumerate}
  \def\labelenumii{\arabic{enumii}.}
  \item
    The MR features \(i < i^*\). By UIP, investors must expect that the
    dollar will appreciate in the future. For this to happen, \(i\) will
    have to rise in the future (by UIP). In this sense, the nice MR
    outcome is not sustainable.
  \item
    A sustainable outcome must stabilize the exchange rate eventually.
    That requires \(i = i^*\), suggesting that the Fed can neutralize
    the foreign monetary shock only for some time. 
  \end{enumerate}
\item
  If the Fed tries to stabilize \(E\), it has to raise \(i\) to \(i^*\)
  (by UIP).

  \begin{enumerate}
  \def\labelenumii{\arabic{enumii}.}
  \item
    Now the MR outcome is \(Y, C\) unchanged. Higher \(i\) crowds out
    \(I\). 
  \item
    Higher \(i\) with unchanged \(Y\) requires that \(M/P\) falls.
  \item
    UIP requires \(E = E^e\). Hence \(E\) is unchanged.
  \item
    \(NX\) must rise to keep AD at \(Y_n\). Hence, the real exchange
    rate must depreciate. 
  \item
    With fixed \(E\), this has to happen through falling prices. AD must
    have shifted \emph{left}. The Fed must \emph{contract} the money
    supply.
  \end{enumerate}
\end{enumerate}

So the overall conclusion is that tighter foreign monetary policy
requires a domestic tightening as well.

\end{document}
